\documentclass[a4paper,10pt]{article}
\usepackage[utf8]{inputenc}
\usepackage{xspace}
\usepackage{graphicx,graphics} 
\usepackage{hyperref} 

\usepackage{algorithm2e}
\SetAlgoLined
\SetKwProg{MyStruct}{Struct}{ contains}{end}



\graphicspath{{img/}}

%opening
\title{Algorithmic model for N-GREEN optical ring}




\begin{document}

\maketitle
\section{Introduction}
Blabla sur le contexte du projet NGREEN



\section{Model}
  \label{model}
  \subsection{Optical ring}
    Graphe oriente representant un anneau, definition des terme delay
  \subsection{Discrete time model}
    temps slotte correspondant a un tick d'horloge
  \subsection{Into the node : Two kind of traffics}
    Descripion buffers/ on prepare des conteneurs.definition des therme paquet slot buffers
    Description CRAN-BE, precision sur le modele de generation
    presentation des differentes politiques d'insertion.

  \subsection{Objectives}
    Etudier Le comportement de nos politiques
    Trouver une solution deterministe et etudier l'impact sur les BE
    
\section{Study of the different insertion policies}

  \subsection{Parameters}
  Peut etre attendre les resultats de youssef pour fixer les deadline et seuil d'envoi dans les simuls.
  \subsection{No management}
  \subsection{CRAN priority}
  Conclure que c'est pas top top, même si c'est clairement mieux en CRAN, et qu'on va faire du determinisme.

\section{Deterministic approach for 0 latency on CRAN}
\label{det}
  \subsection{Packing}
  \subsection{Spreading}
  Pour ces deux la, ecrire proprement les ``algos''
  \subsection{Impact on the BE}
  REsultats de simuls
  
  
\section{ Implementation details}

	The source code int C of the simulator used for our experimental results is available on the \href{https://github.com/Mael-Guiraud/Ngreen.git}{ author's github page}.
	\subsection{Initialisation}
	First, we need to put coherent parameters , considering the model. For instance, with the parameters of NGREEN, we can not set more than 10 antennas on the ring. 
	The ring is modeled by a circular table of struct packets. 
	
	\begin{algorithm}[H]
	\MyStruct{packet}{
 		int owner\_id\; \tcp{-1 if there is no packet on the slot.}
  		int size\_CRAN\; \tcp{Nb messages CRAN in the packet\, useful to generate the answers.}
  		int reservation\_id\; \tcp{Useful in the reservation model\, -1 otherwise.}
	}
	\end{algorithm}
	
	For each antenna, we initialize the first date $t$ on which the antenna send some traffic. The antenna will then send a message every 10 slots after $t$, during a fixed total emission time, depending of the throughput of the antenna.
	
\begin{center}   

      \includegraphics[width=\textwidth]{emission_antenna.pdf}

  
\end{center}

	 In the Insertion Policy models, this date $t$ is randomly drawn with the rand() functions of stdlib. In the Reservation model, this date $t$ is set by the scheduling algorithms described in sec~\ref{det}.
	
	Since the time is discretized, a slot (a unit of time) correspond to one round of a loop. The duration of the loop is set to be large enough to obtain some confident experimental results. During each round of the simulation loop, we first generate the best effort and CRAN traffic, following the laws described in sec~\ref{model}.
	
	Then, we call the \texttt{insert\_packets()} function to send the packets on the ring, if needed (i.e, if some CRAN has been generated, or the BE buffer is large, or old enough).
	Finally, the function \texttt{rotate\_ring()} is called, and every struct packets of the circular ring takes the following position in the table (the last cell of the table take the position 1).
	Afterwards, we call the \texttt{remove\_packets()} function, that free the reading\_slot of a node, if the packet on it has been sent by this node.
	\subsection{BE-Generation}
	The best effort generation follows the SBBP decribed in sec~\ref{model}. Since the batch process have, at the most 10 states, this functions runs in $O(1)$: one random drawing for the transition in the markov chain, another random drawing for the batch process, and the Inverse transform sampling, which is bounded by the number of states.
	\subsection{CRAN-Generation}
	We splitted the CRAN-Generation in two steps. First, the nodes which are not the BBU adds their messages from antennas to buffers, if needed. In a second time, the BBU generates and answer in it's buffer, if there is a CRAN packet in the reading\_slot of the BBU.
	\subsection{Insertion packet}
	A node emits a packet on the ring, in its writting\_slot, if :
\begin{itemize}
\item In no management mode :  if the size of the buffer is greater or equal to $\beta .Q $ ($\beta \in ]0;1] , Q$ is the size of the buffer) , or if the oldest datas has arrived in the buffer before or to $t-\alpha$, ( $t$ is the current slot, and $\alpha$ is the deadline, expressed in slot).
\item In CRAN priority mode : 
	\begin{itemize}
	\item If the BE-buffer satisfy one of the latter condition,
	\item Or if there is some CRAN in the CRAN\_buffer. The packet emitted contains all the CRAN possible, filled with as much BE as possible. 
	\end{itemize}
\item in reservation mode, the slot on the writting\_slot of the node must be not reserved, or reserved for the node. Then the constraints are the same than the CRAN-priority mode.
\end{itemize} 	
\subsection{Slot Reservation}
In reservation mode, the main challenge is to reserve the slots for the nodes. Indeed, if a nodes $i$ needs the slot in his wrtting\_slot at time $t$, this slot needs to be reserved at time $t - size\_ring$ by $i$. In this situation, if the slot is already taken by a packet which has been sent by the node $k$, $k$ will remove the packet of the slot during the next lap of the ring, and no other nodes are able to write in this slot before it cames back to the node $i$ a date $t$.

One must study the impact of the reservation on the behavior of the ring. If we want to send a packet in a slot at time $t$, as we described upper, the slot must be reserved one $size\_ring$ before, by the nodes which needs to send the packet. This is the job of the reservation : avoiding others nodes to write in a slot which will be used by the node $i$ at time $t$. Though his reservation allows the CRAN messages to have 0 latency, it impacts the best effort, by avoiding them to use free slots; For the best efforts packets, a slot full represents the same thing as a reserved slot.

Another behavior of a packet on the ring that we must consider is the following one:
 if a node $i$ sends a packet at time $t$, the slot will be used during $t+ size\_ring$ slots. Thus, even if a node emits once, we must take into consideration the fact that the slot is used in the entire ring during $size\_ring$ slots. This means that no others nodes can use this slot to emit a packet on the ring.
This case is problematic if we have $ET =P/2$. For instance, if we want to send a periodic message of size $50$ each $100$ slots on two nodes on a ring, If the node $u$ sends the first packet of it's message at date $0$. Thus, $u$ will emit some packets every periods, during $100$ slots, that is $[kP;ET+kP[, k \in N$. Then, the node $v$ will receive the messages from $u$ each period at time $[kP + \lambda(u,v);ET+kP+\lambda(u,v)[$. Thus, $b$ will emit on the ring at time $[ET + kP + \lambda(u,v);ET+kP+\lambda(u,v) + ET[$, and $a$ will receive those packets from $b$ to $[ET + kP + \lambda(u,v) + \lambda(v,u);ET+kP+\lambda(u,v) + ET + \lambda(v,u)[$, that is $[ET + kP + RS; (k+1)P+RS[$. If we take $k=0$, then $a$ will emit at time $[0;ET[$ then $[P;ET+P[$, and receive the first message of $b$ at time $[ET + RS;  P+RS[$. He one can observe that, during $RS$ slots after $ET$ in $a$, the ring is used by $b$ and then $a$ cannot emit on the ring. Thus, the maximum size that two nodes can share on the ring is $P/2 -RS$.

We distinguish tree cases for the reservation model:
\begin{enumerate}
\item If $2n < EP $.
\item If $2n >EP . \frac{P}{ET+RS}$
\item if $ET = P/2$
\end{enumerate}
	
In the first case, since we have enough frequencies to order the antennas and theirs answers, we only give one block of frequency to each antennas. For instance, if we have $EP = 10$ and $3$ antennas, we give frequencies $0$ and $1$ to the first antenna, $2$ ans $3$ to the second antenna, and $4$ and $5$ to the third antenna.

One can use a smarter method to order k antennas on n frequencies. By expressing $n = q.k + r$, we put a gap of q frequencies between each antennas, except the last r antennas that are spaced by $q+1$ frequencies.

In the second cases, each bloc of $\frac{P}{ET+RS}$ behave like the first case, and we split the different antennas between the different blocks by using the repartition algorithm described above.

In the third case, if $ET = P/2$, we can not schedule the entire emission of a message on the same frequency, because there will be $RS$ slots during while there will be some conflict at the insertion.
Thus, we need to reserve some frequencies to carry the last packet of the messages.
The number of frequencies reserved is $2 .\lceil \frac{2.RS.N}{P}\rceil$.

	
The application we study takes place in the C-Ran context. We want to centralise calculations into a common data-center for an area of antennas distributed on the territory. The data between the RRH (antennas) and the BBU (data-center) are transiting through an optical ring. We study here only the behaviour of the ring. First, for an easier model, we consider a ring in which one or two nodes are the data-centers, and the other nodes of the ring sends some data coming from the RRH.

Every millisecond, between each couple RRH-BBU, the following periodic process is observed:
\begin{enumerate}
 \item The RRH sends some latency critical data to its BBU.
 \item Those data are transferred to the BBU through the ring .
 \item After a computation time of the message, the BBU sends an answer to its RRH.
\end{enumerate}
 This periodic process has a critical end to end latency, thus a messages has to be as little buffered as possible before being sent in the ring.
 Every messages will browse the entire ring during the process. Since the ring is approximately 50km long, a messages needs $250 \mu s$ to travel this length.
 Considering that the periodicity of the emissions ($1ms$) is greater than a ride of the ring, we take $P$ a multiple of the time needed to travel the ring.
 
 
 
\section*{Sending policy}
In the studied model, the nodes of the ring have a {\bf broadcast and select} policy. Then, the physical times needed to insert the data in the ring are negligible in regard of the other times.
When a node sends a packet on the ring, it contains data for several destinations. Every node of the ring reads all the slots and keep the data addressed to it. It copy the data of the packet in the slot, but does not remove anything. The only node that can free a slot is the one which send a packet in it.

%when a node sends some data in a slot, this slot is {\em taken} by this node for an entire ride of the ring. This means that every nodes of the ring can collect the information addressed to it, but it does not delete the data in the slot. The slot is fully cleared by the node that sent it, when it came back after a ride of the ring.


Thus, a node has to wait for a free slot before sending a packet in the ring. To avoid overloading the ring, a node sends a packet only if some criteria  are satisfied.
We define by $Q$ the capacity of the buffer of a node, $\alpha$ the {\em minimum load} of the buffer, $0 \leq  \alpha \leq 1$, and $\delta$ the {\em maximum delay}.

The node sends a packet in a slot if:
\begin{itemize}
 \item The amount of data $q$ in the buffer is greater than $\alpha . Q$,
 \item or the older data in the buffer is waiting for more than $\delta$.
\end{itemize}

The buffer of a node is filled with the FIFO rule by a traffic coming from external sources. Then, when the node has an available free slot, the packet is inserted in the slot if one of the previous constraints are satisfied. Otherwise the node does not send anything in the ring and wait for the next free slot.

\begin{center}   

      \includegraphics[scale=0.7]{insertion0.pdf}

  
\end{center}

\section{C-RAN context}

In C-RAN context, this amount of data $q$ contains two kinds of data: the latency critical flow coming from RRH or BBU, and an other flow, called {\em best effort}.

We can consider two buffers containing $x$ and $y$ amount of data. The first one is filled with the C-Ran data in priority, and the best effort flow if needed, the second one takes the rest of best effort flow. We consider that, even if a lot of RRH are represented by a node on the ring, the amount of C-Ran data incoming to a node is strongly lower than the capacity of a packet ($\simeq 1.2$ Gbps for each RRH, 100 Gbps into the ring).
The C-RAN data comes periodically in regard of the period $P = 1ms$. 
The best effort data comes more randomly (can be modelled by a Poisson distribution, for example), and are inserted in the buffer if $x = 0$.

\begin{center}   

      \includegraphics[scale=0.7]{insertion1.pdf}

  
\end{center}
 
 
\section*{Model}
We consider an optical ring composed of $k$ nodes ($k={3,...,10}$).
We are in a discrete time model: the time is split in $N$ slot, rotating to the following position each time step $t = 10 \mu s$. This value is the switching granularity.
\begin{center}   

      \includegraphics[scale=0.5]{anneau1.pdf}
      \hspace{3cm}
      \includegraphics[scale=0.5]{anneau2.pdf}
  
\end{center}

Considering a length of 50 km for the ring, we evaluate the number of slots about 25 into the ring.

\section*{Study}
The purpose is here to reduce the latency of the C-RAN data. Two event can increase this latency: 
\begin{enumerate}
  \item The node has enough data $q$ to emit in the ring, but no slots are available.
 \item The node does not have enough data $q$ to emit and a data from $x$ is waiting for less than $\delta$.
\end{enumerate}

In a first time, we will look at the first problem, and try to determine from what load of the ring, the time taken to wait a free slot in the ring is too long.



\end{document}