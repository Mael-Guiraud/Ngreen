\documentclass[a4paper,10pt]{article}
\usepackage[utf8]{inputenc}
\usepackage{xspace}
\usepackage{graphicx,graphics} 
\usepackage{color}
\usepackage{amsmath}
\usepackage{amsfonts}
\usepackage{amssymb}
\usepackage{amsthm}
\usepackage{algorithm}
\usepackage{algorithmic}
\usepackage{longtable}
\usepackage{complexity}
\usepackage{tkz-graph}
\usepackage{float}
\usepackage{setspace}
\renewcommand{\algorithmicrequire}{\textbf{Input:}}
\renewcommand{\algorithmicensure}{\textbf{Output:}}
  
\graphicspath{{figures/}}
\newcommand\rmatching{${\cal R}$-matching\xspace}
\newcommand\mdelay{$\cal M$-delay\xspace}
\newcommand\matchedgraph{{\bf matched graph}}
\newtheorem{proposition}{Proposition}
\newtheorem{theorem}{Theorem}
\newcommand{\reporttitle}{Contention management for Deterministic Networking}     % Titre
\newcommand{\reportauthor}{Maël \textsc{Guiraud }} % Auteur
\newcommand{\reportsubject}{Master Thesis} 
\newcommand{\HRule}{\rule{\linewidth}{0.5mm}}
\setlength{\parskip}{1ex} % Espace entre les paragraphes


\newcommand{\todo}[1]{{\color{red} TODO: {#1}}}


%opening
\title{Contention Management for 5G}
\author{DB,CC,MG,OM,YS}


\begin{document}

\maketitle

\begin{abstract}
This article treats about Contention Management for 5G.
\end{abstract}

\section{Introduction}
  \itemize
    \item Context and problematic
    \item Related works
    \item Article contribution

\section{Model, Problems}

  \subsection{Definitions}
  
	We consider a symmetric directed graph $G=(V,A)$ modelling a network. Each arc  $(u,v)$ in $A$ is labeled by an integer $Dl(u,v) \geq 1$ that we call the delay and
	which represents the number of time slots taken by a signal to go from $u$ to $v$ using this arc. Note that for any arc $(u,v)$, $Dl(u,v)=Dl(v,u)$.
	
      A {\bf route} $r$ in $G$ is a sequence of consecutive arcs $a_0, \ldots , a_{k-1}$, with $a_i=(u_i,u_{i+1}) \in A$. 
      
      The {\bf latency} of a vertex $u_i$ in $r$, with $i \geq 1$, is defined by $$\lambda(u_i,r)= \sum\limits_{0 \leq j <i} Dl(a_j)$$ We also define $\lambda(u_0,r)=0$.
      The latency of the route $r$ is defined by $\lambda (r)= \lambda (u_k,r)$.
      
      A {\bf routing function} $\cal R$ in $G$ associates to each pair of vertices $(u,v)$ a route from $u$ to $v$. Let $\cal C$ be an {\bf assignment} in $G$, i.e., a set of couples of different vertices of $G$. We denote by $\cal R_{\cal C}$ the set of routes ${\cal R}(u,v)$ for any $(u,v)$ in $\cal C$. \todo{Est-ce qu'on doit ajouter que le routage est cohérent ?}

   \subsection{Slotted time Model}
      Consider now a positive integer $P$ called the {\bf period}. A {\bf $P$-periodic affectation} of $\cal R_{\cal C}$ consists in a set  ${\cal M}=(m_0, \ldots ,m_{c-1})$
      of $c$ integers that we call {\bf offsets}, with $c$ the cardinal of $\cal C$. In our problem the messages we send in the network will be periodic of period $P$ and thus we consider slices of time of $P$ slots. 
      The number $m_i$ represents the number of the first slot used by the route $r_i \in {\cal R}_{\cal C}$ at its source in a period.
      We define the first time slot at which a message reaches any vertex $v$ in this route by $$t(v,r_i) = m_i+\lambda(v,r_i) \mod P.$$

      A message usually cannot be transported in a single time slot. We denote by $\tau$ the number 
      of slots necessary to transmit a message. %remarque sur le fait qu'un message est porté par des temps consécutifs ?
      Let us call $[t(v,r_i)]$ the index of the time slots used by a route $r_i$ at a vertex $v$ in a period $P$. Those values are forming a consecutive set of values starting at $t(v,r_i)$ and ending at $t(v,r_i) + \tau \mod P$. A $P$-periodic affectation must have no {\bf collision} between two routes in ${\cal R}_{\cal C}$, that is $\forall (r_i, r_j) \in {\cal R}_{\cal C}^2, i \neq j$, % with $\tau$ the size (in number of consecutive slots) of each message that must be periodically sent on each route of ${\cal R}_{\cal C}$, 
      we have $$[t(u,r_i)] \cap [t(u,r_j)] = \emptyset .$$
      
   \subsection{Problems}

   \todo{Mettre une phrase ou deux pour faire le lien avec le problème concret, plus facile  faire une fois l'intro écrite}
      The main theoretical problem we have to deal with in this context is the following.\\

      \noindent {\bf Problem  Periodic Routes Assignment (PRA)} 

      \noindent {\bf Input:} a graph $G=(V,A)$, a set $\cal C$ of pair of vertices, a routing function $\cal R$ and an integer $P$.

      \noindent {\bf Question:} does there exist a $P$-periodic affectation of $\cal C$ in $(G,{\cal R})$?

      We deal in next section with the complexity of the Problem PRA.\\
% 
%       \begin{figure}[H]
%       \label{could-ran}
%       \begin{center}
%       % \begin{tabular}{cc}
%       \includegraphics[scale=0.5]{Total-latence.pdf}
%       \caption{Complete process for a leaf in $L$.}
%       \end{center}
%       \end{figure}
%       %\end{tabular}\newline

      
      In the context of cloud-RAN applications, we consider here the digraph $G=(V,A)$ modeling the target network 
      and two disjoint subsets of vertices $S$ and $L$, where $S$ is the set of BBU and $L$ is the set of RRH. 
      We denote by $n$ the size of $S$ and $L$.
      We are given a period $P$, a routing function ${\cal R}$ and a bijection $\rho:L\rightarrow S$ which defines two disjoint assignments ${\cal C}_1 = \{(l,\rho(l))\}_{l \in L}$ and ${\cal C}_2 = \{(s,\rho^{-1}(s))\}_{s \in S}$. Let ${\cal M} = (m_1,\dots, m_n)$ and ${\cal W} = (w_1,\dots, w_n)$ be two $P$-periodic affectations, of respectively ${\cal R}_{{\cal C}_1}$ and ${\cal R}_{{\cal C}_2}$.
      
      The process realized periodically (i.e. initiated in each window of size $P$) for each leaf $l \in L$ is the following one (see Figure \ref{cloud-ran}). First, a message of $\tau$ slots is sent from $l$ to $\rho(l)$ on ${\cal R}(l,\rho(l))$ at slot $m_l$ in the current window. After receiving this message, $\rho(l)$ computes it during a time equal to $\theta$ slots.\todo{Je suis contre
      le theta dans le modèle, car il ne set à rien. On peut dire par contre dans le blabla autour du modèle qu'on simule le temps de calcul par des arêtes plus longues}
    
      Then, a message of $\tau$ slots containing the computed result is sent back from $\rho(l)$ to $l$ on ${\cal R}(\rho(l),l)$, after waiting for $w_l$ slots. 
      %at the first occurrence of  step $m_{\overline l}$ in a window (i.e., in the current window at the end of computation or the next one). We denote by $\omega(l)$ the {\bf waiting time} of $l$, i.e.,  the number of slots between the end of the computation time in $\rho(l)$ and the first occurrence of  step $m_{\overline l}$ in a window. 
      Thus, the whole proccess time for $l$ is equal to
      $$
      PT(l)=\lambda({\cal R}(l,\rho(l)))+\theta+w_l+\lambda({\cal R}(\rho(l),l)).
      $$
      
      \todo{Il manque le fait qu'il n'y a pas de collisions ni à l'aller ni au retour.}
% 	
%       Let us denote by ${\cal C}_{\rho}$ the set of couples $<l,\rho(l)>$ and $<\rho(l),l>$, for any $l \in L$. Consider a $P$-periodic affectation ${\cal M}$ of ${\cal C}_{\rho}$ in $(G,{\cal R})$. 
    The {\bf maximum process time} of $({\cal M},{\cal W})$ is defined by $MT(({\cal M},{\cal W}))=\max\limits_{l \in L} PT(l)$. The problem we want to solve, is the following. 

      \noindent {\bf Problem Periodic Assignment for Low Latency(PALL)} 

      \noindent {\bf Input:}  a digraph $G$, a matching $\rho$ of a set $S$ into a set $L$ in $G$, a routing function $\cal R$, a period $P$, an integer $T_{max}$.

      \noindent {\bf Question:} does there exist a pair of $({\cal M},{\cal W})$ $P$-periodic affectation ${\cal M}$ of ${\cal C}_{\rho}$ in $(G,{\cal R})$ such that $MT({\cal M}) \leq T_{max}$?

      %The related optimisation problem we will focus on  consists in minimizing  $MT({\cal M})$. Note that in the context of cloud-RAN networks, we consider $P=1ms$, $\theta=2.6ms$ and $T_{max}$ must be less or equal to $3ms$.
      %cette remarque doit être dans la partie expérimentale


	

  
\section{PRA Solving}
  
  \subsection{NP-Hardness}
	
   \todo{Definition of the load. La preuve se réfère à la précédente qui n'a pas été retenue, il faut du coup tout expliquer.}
   
   
    \begin{theorem}
    Problem PRA cannot be approximated within a factor $n^{1-o(1)}$ unless $\P = \NP$ even when the load is two
    and $n$ is the number of pairs in the assignment.
    \end{theorem}

    \begin{proof}
    We reduce PRA to graph coloring. Let $G$ be a graph instance of the $k$-coloring problem. 
    We define $H$ in the following way: for each vertex $v$ in $G$, there is a route $r_v$ in $H$.
    Two routes $r_v$ and $r_u$ share an edge if and only if $(u,v)$ is an edge in $G$ and this edge is only in this two routes. 
    We put a weight inbetween shared edges in a route so that there is a delay $k$ between two such edges. 
    
    As in the previous proof, a $k$-coloring of $G$ gives a $k$-periodic schedule of $H$
    and conversly. Therefore if we can approximate the value of PRA  within a factor $f$,
    we could approximate the minimal number of colors needed to color a graph within a fator $f$, 
    by doing the previous reduction for all possible $k$. The proof follows from the hardness of approximability
    of finding a minimal coloring~\cite{zuckerman2006linear}.
    \end{proof}


   
  \subsection{MIN-PRA}
    Exemple de cas simple
    
\section{Proposed Solutions, solving PALL}
  
  In this section, we consider a particular case of the model, in which for each $(u,v)$ , the route is the same in both directions. This means that ${\cal R}(u,v)$ uses the same arcs as ${\cal R}(v,u)$ in the opposite orientation.
  \subsection{Intro}
    PALL NP-Hard car PRA NP-Hard\\
    Résultats valables sur Topologie 1 avec nos paramètres
    \todo{J'ai viré star affectation, car je pense qu'il n'y a rien à dire là dessus.}
    
  \subsection{No waiting times}
    \subsubsection{Shortest-longest}
      \paragraph{Algo}
      \paragraph{Period}
    \subsubsection{Exhaustive generation}
      Décrire l'algo, expliquer les coupes
    \subsubsection{Results}
      Resultats des simulations : Shortest-longest optimal pour ces parametres.
      
   \subsection{Allowing waiting times}
     \subsubsection{Intro}
	Importance des waiting times quand la période est donnée (Résultats D'éxepriences et preuve avec l'exemple)
     \subsubsection{LSG}
	\paragraph{Algorithm}
	\paragraph{Analysis}
	  Parler de LSO et expliquer pourquoi LSG mieux avec nos params
     \subsubsection{Results}
	 \paragraph{Random}
	 \paragraph{Distributions}
   
\section{Conclusion}

\end{document}
